\section{Algorithme de recombinaison}
\subsection{Spécification de l'algorithme}
L'algorithme de recombinaison a pour objectif de maximiser le nombre de recettes réussies par batch toute, respectant les tolérances sur les quantités demandées.
 Pour chaque batch, comprenant environ $300$ recettes, il doit sélectionner les microcapsules disponibles dans le stock,
 en s'adaptant aux contraintes suivantes :
\begin{itemize}
    \item les capsules disponibles peuvent ne pas convenir à toutes les recettes,
    \item les quantités demandées doivent être respectées avec une tolérance de $\pm 10\%$,
    \item chaque réacteur peut contenir un maximum de $5$ capsules.
\end{itemize}
L'algorithme doit :
\begin{enumerate}
    \item Optimiser le nombre de recettes réalisées dans le batch.
    \item Minimiser l'écart entre la quantité déposée et la quantité demandée pour chaque recette.
    \item Minimiser les manipulations de plaques.
\end{enumerate}
Il possède, en entrée : 
\begin{itemize}
    \item les recettes du batch,
    \item les microcapsules disponibles dans le stock.
\end{itemize}

La sortie contient : 
\begin{itemize}
    \item les plaques de stockage à utiliser,
    \item la disposition des microcapsules sur les plaques,
    \item les réacteurs dans lesquels les microcapsules doivent être placées,
    \item les recettes non réalisables.
\end{itemize}
\subsection{Méthode d'optimisation et stratégie d'ordonnancement}
\subsection{Interaction entre le logiciel et le matériel pour la recombinaison}