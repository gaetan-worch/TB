\section{SwissCat+}
Les catalyseurs est une espèce chimique qui permet ou accélère la réaction chimique sans être consummé dans le processus. Les catalyseurs sont utilisées dans divers domaines (Agricole, militaire, chimie, traitements des déchets, transformation de polluants, \dots). Ils sont indispensables dans notre société, cependant, ces produits requièrent souvent des terres rares. Par exemple, les catalyseurs dans les voitures (qui servent à réduire les émissions de gaz polluant en transformant notamment le monoxyde de carbone, les hydrocarbures imbrûlés et de l'oxyde d'azote en eau, dioxide de carbone et dioxide d'azote) sont composée d'alumine, d'oxyde de cérium mais surtout ils sont composés d'au moins trois platinoïde.


C'est dans ce contexte de réduction d'utilisation de terre rare que le laboratoire SwissCat+ a été crée dans le but d'optimiser la composition des catalyseurs et d'en trouver des nouveaux. Le laboratoire est scindé en deux, le laboratoire Est, située à l'ETHZ à Zurich, s'occupe de faire des recherches sur les catalyseurs hétérogène, tandis que la partie du laboratoite Ouest, situé à l'EPFL à Lausanne, fera des recherches sur les catalyseurs homogènes.
\subsection{laboratoire ouest}
Le laboratoire doit être automatiser afin de pouvoir de grandes quantités d'expériences afin de pouvoir explorer un maximum de l'espace chimique. Pour avoir un espace chimique le plus grand possible, il faut que le laboratoire puisse manipuler des matières liquides et de larges quantité de poudre ($\qtyrange[range-units=single]{0.1}{50}{\mg} $).

\section{Contexte et objectifs du projet Storm}
C'est pour pouvoir gérer cette plage de quantité de poudre que le projet STORMS (\textit{STOchasitic Robotized Micro Sampling}), (en collaboration entre l'EPFL, l'HEIG-VD, Chemspeed Technologies et Dietrich Engineering Consultants) est né. STORMS est composés de plusieurs modules :
\begin{itemize}
    \item la standardisation,
    \item le stockage,
    \item le micro-échantillonage (\og microsampling\fg),
    \item la recombinaison,
    \item la box d'Edy
    \item les box de Chemspeed
\end{itemize}
La standardisation à pour but de donner des récipients de taille standard aux microsampling qui va créée des microcapsules avec des quantités qui cible les quanntités souhaitée qui sont stockées sur des plaques nommées \og \textit{wellplate}\fg, qui sont ensuite rangées dans sur des rack dans le stockage. Ce stockage sert de lien entre la standardisation, le \og microsampling\fg  et la recombinaison.
La recombinaison doit déterminer quelles microcapsules choisir pour toutes les recettes présentent dans le \og \textit{batch}\fg.

\section{Recombinaison de microcapsules}
Le processus de recombinaison consiste à assembler différentes microcapsules de poudres pour répondre aux besoins expérimentaux. Le logiciel de recombinaison sélectionne les capsules
disponibles afin d'obtenir une composition approchant les quantités nécessaires pour une expérience donnée. Le but est de maximiser le nombre d'expériences réalisées.