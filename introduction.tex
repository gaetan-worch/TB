\section{SwissCat+}
Les catalyseurs est une espèce chimique qui permet ou accélère la réaction chimique sans être consummé dans le processus. Les catalyseurs sont utilisées dans divers domaines (Agricole, militaire, chimie, traitements des déchets, transformation de polluants, \dots). Ils sont indispensables dans notre société, cependant, ces produits requièrent souvent des terres rares. Par exemple, les catalyseurs dans les voitures (qui servent à réduire les émissions de gaz polluant en transformant notamment le monoxyde de carbone, les hydrocarbures imbrûlés et de l'oxyde d'azote en eau, dioxide de carbone et dioxide d'azote) sont composée d'alumine, d'oxyde de cérium mais surtout ils sont composés d'au moins trois platinoïde.


C'est dans ce contexte de réduction d'utilisation de terre rare que le laboratoire, SwissCat+, a été crée dans le but d'optimiser la composition des catalyseurs et d'en trouver des nouveaux. Le laboratoire est scindé en deux, le laboratoire Est, située à l'ETHZ à Zurich, s'occupe de faire des recherches sur les catalyseurs hétérogène, tandis que la partie du laboratoire Ouest, situé à l'EPFL à Lausanne, fera des recherches sur les catalyseurs homogènes.
\subsection{Laboratoire ouest}
Le laboratoire doit être automatiser afin de pouvoir effectuer de grandes quantités d'expériences pour explorer un maximum l'espace chimique. Pour avoir un espace chimique le plus grand possible, il faut que le laboratoire puisse manipuler des matières liquides et de larges quantités de solides ($\qtyrange[range-units=single]{0.1}{50}{\mg} $).

\section{Contexte et objectifs du projet Storm}
Le projet STORMS (\textit{STOchasitic Robotized Micro Sampling}) est né d'une collaboration entre l'EPFL, l'HEIG-VD, Chemspeed Technologies et Dietrich Engineering Consultants (DEC) afin de pouvoir manipuler cette plage de quantité de solides.

STORMS est composés de $7$ modules (cf. \autoref{fig:schema_module_storms}):
\begin{itemize}
    \item la standardisation,
    \item le stockage,
    \item le micro-échantillonage (\textit{microsampling}),
    \item la recombinaison,
    \item la synthBox
    \item $2$ box de Chemspeed
\end{itemize}
\begin{figure}[h]
    \centering
    \includegraphics[options]{assets/figures/schema_storms.drawio}
    \caption{Schema des modules de STORMS}
    \label{fig:schema_module_storms}
\end{figure}
La standardisation a pour but de donner des récipients de taille standard aux microsampling qui va créer des \gls{microcapsule}.
Les \gls{microcapsule} peuvent contenir de $\qtyrange[range-units=single]{0.1}{10}{\mg}$ de produit. Dû aux grandes différences de caractéristique des produits à expérimenter, il n'est pas possible de déterminer précisement à l'avance la quantité qui sera délivrée dans les \gls{microcapsule} par le \textit{microsampling}.
Les \gls{microcapsule}, une fois fabriquées, sont pesées puis mises sur plaques nommées \gls{wellplate}, qui sont ensuite stockée Le stockage fait lien entre la standardisation, le \textit{microsampling} et la recombinaison.
La recombinaison doit réaliser les recettes voulues à partir du stock disponible.

\section{Recombinaison de microcapsules}
Le processus de recombinaison consiste à assembler différentes \gls{microcapsule} de poudres pour répondre aux besoins expérimentaux. Le logiciel de recombinaison sélectionne parmi les \gls{microcapsule} disponibles afin d'obtenir une composition entrant dans les tolérances quantités pour une expérience donnée. Le but est de maximiser le nombre d'expériences réalisées. Une fois les \gls{microcapsule} sélectionnées, la \gls{glovebox} recombinaison doit placer ces \gls{microcapsule} dans les réacteurs correspondant aux recettes réalisées.