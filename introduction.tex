
\section{Contexte et objectifs du projet Storm}
Le projet STORMS (\textit{STOchasitic Robotized Micro Sampling}) vise à automatiser la manipulation de très petites quantités de poudre dans des capsules en verre,
répondant aux besoins de la recherche en chimie et en science des matériaux. Ce projet, financé par Innosuisse et réalisé en collaboration avec l'EPFL, l'HEIG-VD, Chemspeed Technologies
et Dietrich Engineering Consultants, cible l'amélioration de la préparation de micro-échantillons pour accélérer la découverte de nouveaux produits chimiques.
\subsection{Méthodologie et défis techniques}
La manipulation de poudres à l'échelle sub-milligramme est complexe en raison des propriétés variées des poudre (taille, densité, morphologie). STORMS propose une approche innovante
de \og{}\textit{sampling stochasitic}\fg{}, générant une bibliothèque de masses aléatoires. Cela permet de contourner la difficulté de peser précisément une masse cible.

\section{Recombinaison de microcapsules}
Le processus de recombinaison consiste à assembler différentes microcapsules de poudres pour répondre aux besoins expérimentaux. Le logiciel de recombinaison sélectionne les capsules
disponibles afin d'obtenir une composition approchant les quantités nécessaires pour une expérience donnée. Le but est de maximiser le nombre d'expériences réalisées.