\documentclass[
    iai, % Saisir le nom de l'institut rattaché
    eai, % Saisir le nom de l'orientation
    %confidential, % Décommentez si le travail est confidentiel
]{heig-tb}

\usepackage[nooldvoltagedirection,european,americaninductors]{circuitikz}
\usepackage{siunitx}
\usepackage{amssymb}
\signature{mbernasconi.svg} % Remplacer par votre propre signature vectorielle.

\makenomenclature
\makenoidxglossaries
\makeindex

\addbibresource{bibliography.bib}

\input{nomenclature}
\input{acronyms}
\newglossaryentry{heig-vd}{
    name=HEIG-VD,
    description={Haute École d'Ingénierie et de Gestion du canton de Vaud}
}
\newglossaryentry{hes-so}{
    name=HES-SO,
    description={Haute École Supérieure de Suisse Occidentale}
}
\newglossaryentry{latex}{
    name=latex,
    description={Un langage et un système de composition de documents}
}
\newglossaryentry{wellplate}{
    name=wellplate,
    text=microplate,
    description=Plaque contenant les microcapsules
}
\newglossaryentry{paradox}{
    name=paradox,
    description=Nom de la boîte contenant les réacteurs
}
\newglossaryentry{microcapsule}{
    name=microcapsule,
    description=Petite capsule en verre servant à contenir des produits chimiques
}
\newglossaryentry{glovebox}{
    name={glove box},
    text={\textit{glove box}},
    description=Boîte scellée permettant de manipuler les objets à l'intérieur grâce à des gants
}
\newglossaryentry{wrapper}{
    name=wrapper,
    text={\textit{wrapper}},
    description=Programme qui englobe un autre programme
}
\newglossaryentry{synthese}{
    name=synthèse,
    description=Une synthèse chimique est un enchaînement de réactions chimiques
}
\newglossaryentry{recette}{
    name=recette,
    description={Liste des produits, avec leur masse et leur tolérance, à mettre dans un réacteur}
}
\newglossaryentry{batch}{
    name=batch,
    text={\textit{batch}},
    description=Liste de recette
}
\newglossaryentry{bruteforce}{
    name={force brute}, 
    text={force brute},
    description={En informatique, la méthode par force brute consiste contrôler toutes les combinaisons possibles d'un problème}
}
% Auteur du document (étudiant-e) en projet de Bachelor
\author{Gaëtan Worch}

% Activer l'option pour l'accord du féminin dans le texte
\genre{male}

% Titre de votre travail de Bachelor
\title{Conception d'un système robotique avancé pour une application de micro pick-and-place}

% Le sous titre est optionnel
\subtitle{Travail de Bachelor}

% Nom du professeur responsable
\teacher {Prof. G. Costanzo (HEIG-VD)}

% Mettre à jour avec la date de rendu du travail
\date{\today}

% Numéro de TB
\thesis{7212}


\surroundwithmdframed{minted}

%% Début du document
\begin{document}
\selectlanguage{french}
\maketitle
\frontmatter
\clearemptydoublepage

%% Requis par les dispositions générales des travaux de Bachelor
\preamble
\authentification

%% Résumé / Résumé publiable / Version abrégée
\begin{abstract}
    \input{abstract}
\end{abstract}

%% Sommaire et tables
\clearemptydoublepage
{
    \tableofcontents
    \let\cleardoublepage\clearpage
    \listoffigures
    \let\cleardoublepage\clearpage
    \listoftables
    \let\cleardoublepage\clearpage
    \listoflistings
    }
    
\printnomenclature
\clearemptydoublepage
\pagenumbering{arabic}

%% Contenu
\mainmatter
\chapter{Introduction}
\section{SwissCat+}
Les catalyseurs est une espèce chimique qui permet ou accélère la réaction chimique sans être consummé dans le processus. Les catalyseurs sont utilisées dans divers domaines (Agricole, militaire, chimie, traitements des déchets, transformation de polluants, \dots). Ils sont indispensables dans notre société, cependant, ces produits requièrent souvent des terres rares. Par exemple, les catalyseurs dans les voitures (qui servent à réduire les émissions de gaz polluant en transformant notamment le monoxyde de carbone, les hydrocarbures imbrûlés et de l'oxyde d'azote en eau, dioxide de carbone et dioxide d'azote) sont composée d'alumine, d'oxyde de cérium mais surtout ils sont composés d'au moins trois platinoïde.


C'est dans ce contexte de réduction d'utilisation de terre rare que le laboratoire, SwissCat+, a été crée dans le but d'optimiser la composition des catalyseurs et d'en trouver des nouveaux. Le laboratoire est scindé en deux, le laboratoire Est, située à l'ETHZ à Zurich, s'occupe de faire des recherches sur les catalyseurs hétérogène, tandis que la partie du laboratoire Ouest, situé à l'EPFL à Lausanne, fera des recherches sur les catalyseurs homogènes.
\subsection{Laboratoire ouest}
Le laboratoire doit être automatiser afin de pouvoir effectuer de grandes quantités d'expériences pour explorer un maximum l'espace chimique. Pour avoir un espace chimique le plus grand possible, il faut que le laboratoire puisse manipuler des matières liquides et de larges quantités de solides ($\qtyrange[range-units=single]{0.1}{50}{\mg} $).

\section{Contexte et objectifs du projet Storm}
Le projet STORMS (\textit{STOchasitic Robotized Micro Sampling}) est né d'une collaboration entre l'EPFL, l'HEIG-VD, Chemspeed Technologies et Dietrich Engineering Consultants (DEC) afin de pouvoir manipuler cette plage de quantité de solides.

STORMS est composés de $7$ modules (cf. \autoref{fig:schema_module_storms}):
\begin{itemize}
    \item la standardisation,
    \item le stockage,
    \item le micro-échantillonage (\textit{microsampling}),
    \item la recombinaison,
    \item la synthBox
    \item $2$ box de Chemspeed
\end{itemize}
\begin{figure}[h]
    \centering
    \includegraphics[options]{assets/figures/schema_storms.drawio}
    \caption{Schema des modules de STORMS}
    \label{fig:schema_module_storms}
\end{figure}
La standardisation a pour but de donner des récipients de taille standard aux microsampling qui va créer des \gls{microcapsule}.
Les \gls{microcapsule} peuvent contenir de $\qtyrange[range-units=single]{0.1}{10}{\mg}$ de produit. Dû aux grandes différences de caractéristique des produits à expérimenter, il n'est pas possible de déterminer précisement à l'avance la quantité qui sera délivrée dans les \gls{microcapsule} par le \textit{microsampling}.
Les \gls{microcapsule}, une fois fabriquées, sont pesées puis mises sur plaques nommées \gls{wellplate}, qui sont ensuite stockée Le stockage fait lien entre la standardisation, le \textit{microsampling} et la recombinaison.
La recombinaison doit réaliser les recettes voulues à partir du stock disponible.

\section{Recombinaison de microcapsules}
Le processus de recombinaison consiste à assembler différentes \gls{microcapsule} de poudres pour répondre aux besoins expérimentaux. Le logiciel de recombinaison sélectionne parmi les \gls{microcapsule} disponibles afin d'obtenir une composition entrant dans les tolérances quantités pour une expérience donnée. Le but est de maximiser le nombre d'expériences réalisées. Une fois les \gls{microcapsule} sélectionnées, la \gls{glovebox} recombinaison doit placer ces \gls{microcapsule} dans les réacteurs correspondant aux recettes réalisées.

\chapter{Problématique de la recombinaison}
\section{Objectifs}
Concevoir et développer une solution permettant le transfert automatisé, précis et rapide de capsules de réactif chimique entre le stockage et les réacteurs. Un algorithme devra être conçu pour optimiser la sélection des capsules afin de minimiser l'erreur entre la quantité de produit demandée et celle déposée. L'ensemble du système devra garantir un haut niveau de fiabilité et de sécurité dans le processus.
\section{Cahier des charges fonctionnel}
\rowcolors{3}{gray!10}{white}
\begin{longtable}{l|m{5cm}|m{5cm}}
    \caption{Cahier des charges fonctionnel}\\
    \hline \multicolumn{1}{c|}{\textbf{fonction}} & \multicolumn{1}{c|}{\textbf{énoncé de la fonction}} & \multicolumn{1}{c}{\textbf{éxigence}} \\ \hline 
    \endfirsthead
    
    \multicolumn{3}{c}%
    {{\textbf{\tablename\ \thetable{}}-- continued from previous page}} \\
    \hline \multicolumn{1}{c|}{\textbf{fonction}} & \multicolumn{1}{c|}{\textbf{énoncé de la fonction}} & \multicolumn{1}{c}{\textbf{éxigence}} \\ \hline 
    \endhead
    \hline \multicolumn{3}{r}{{Continued on next page}} \\ \hline
    \endfoot
    \hline \hline
    \endlastfoot
    FP $1$&\centering Manipuler des microcapsules de manière automatisée, sans les endommagées&\begin{itemize}
            \item Ne pas détériorer la microcapsule
        \end{itemize}\\
        FP $1.1$&\centering Prélever les microcapsules dans une plaque& \begin{itemize}
            \item Position de prise arbitraire
            \item Contrôler que la microcapsule soit saisie
        \end{itemize}\\
        FP $1.2$&\centering Déplacer les microcapsules&\\
        FP $1.3$&\centering Déposer les microcapsules&\begin{itemize}
            \item Position de dépose arbitraire dans la plaque de réacteurs
        \end{itemize}\\
        FP $2$&\centering Déterminer les microcapsules les plus adaptées pour chaque réacteur, selon une recette donnée&\\
        FP $2.1$&\centering Recevoir la recette pour chaque réacteur&\\
        FP $2.2$&\centering Accéder à la base de donnée du stock&\\
        FP $2.3$&\centering Déterminer la combinaison de microcapsules optimal pour délivrer la masse de produit donnée&\begin{itemize}
            \item Précision dans la masse délivrée : varie à chaque recette.
            \item Nombre maximale de microcapsules par réacteur : $5$
        \end{itemize}\\
        FP $2.4$&\centering Transmettre la position des microcapsules dans le stock et sur la plaque&\\
        FP $2.5$&\centering Informer le stock des microcapsules prélevées&\\
        FC $1$&\centering Respecter les dimensions de l'endroit confiné &Dimension de la boîte  : $\numproduct{133 x 95 x 94}~\unit{\cm}$\\
        FC $2$&\centering Utiliser les énergies disponibles& \begin{itemize}
            \item Électrique : \begin{itemize}
                \item $\qty{400}{\volt}$ triphasé
                \item $\qty{230}{\volt}$ monophasé
            \end{itemize}
            \item Pneumatique : $\qty{8}{\bar}$
        \end{itemize}\\ 
        FC $3$&\centering Utiliser les plaques déjà présentes&\begin{itemize}
            \item Nombre de trou sur la plaque de prise : $384$
            \item Nombre de trou sur la plaque de dépose : $48$
        \end{itemize}\\
        FC $4$&\centering S'adapter aux éléments déjà présents&\begin{itemize}
            \item API : Beckhoff
        \end{itemize}\\
        FC $5$&\centering Dimension des microcapsules : $\varnothing~\qty{3}{\mm}$&

    \end{longtable}
\rowcolors{0}{white}{white}
\chapter{Hardware}
\section{Défis techniques du Hardware}
Le hardware doit pouvoir manipuler des plaques de microcapsules ou de réacteurs, ainsi que des microcapsules en verre de $\qty{3}{\mm}$ de diamètre en garantissant leur intégrité. Le tout doit être dans un espace confiné.
\section{Recherches de solution}
Pour la recherche des solutions, le hardware a été décomposé par les fonctions suivantes : 
\begin{itemize}
    \item Saisie et dépose des microcapsules;
    \item déplacement des microcapsules;
    \item saisie et dépose des plaques.
\end{itemize} 
La position des plaques dans la glove box a également été étudié avec $6$ configurations différentes.
\subsection{Saisie et dépose des microcapsules}
Pour la saisie des microcapsules, les grandes familles de solutions proposées sont : 
\begin{itemize}
    \item Aspiration;
    \item Mécanique; \begin{itemize}
        \item Pince à doigt;
        \item Pince Gecko.
    \end{itemize}
\end{itemize}
\subsubsection*{Avantage et inconvénients}
\begin{table}[H]
    \caption{Avantages et inconvénients des solutions de saisie des microcapsules}
    \begin{tabular}{@{}cll@{}}
    \toprule
    Solution      & \multicolumn{1}{c}{Avantages}                                                                                                   & \multicolumn{1}{c}{Inconvénient}                                                                                                                                  \\ \midrule
    Aspiration    & \begin{tabular}[c]{@{}l@{}}- Exerce moins de pression directe \\ - Simple à installer\\ - Nécessite peu d'entretien\end{tabular} & \begin{tabular}[c]{@{}l@{}}- Apporter l'énergie pneumatique\\ - Bruyant\end{tabular}                                                                     \\
    Pince à doigt & \begin{tabular}[c]{@{}l@{}}- Contrôle précis\\ - Faible coût\end{tabular}                                                        & \begin{tabular}[c]{@{}l@{}}- Maintenance fréquente\\ - Ne convient pas au petits objets\\ - Espace limité, pour pouvoir ouvrir \\ et fermer la pince \\ - Nécessite un contrôle de force\end{tabular} \\
    Pince Gecko   & \begin{tabular}[c]{@{}l@{}}- Saisie non intrusive\\ - Ne nécessite pas de source \\ d'énergie externe\end{tabular}                            & \begin{tabular}[c]{@{}l@{}}- Capacité de charge\\ - Nécessite un nettoyage pour maintenir \\  l'adhérence\\ - Détachement complexe\end{tabular}     \\ \bottomrule
    \end{tabular}
\end{table}
\subsection{Déplacement des microcapsules}
Pour le déplacement des microcapsules de leur plaque jusqu'aux réacteurs, trois idées ont été étudiées : 
\begin{itemize}
    \item Transport pneumatique par tube;
    \item convoyeur;
    \item robot.
\end{itemize}
\subsection*{Transport pneumatique par tube}
Le système de transport pneumatique par tube\footnote{\href{https://fr.wikipedia.org/wiki/Tube_pneumatique}{Tube pneumatique - Wikipédia}}, serait des tuyaux dans lesquelles naviguent les microcapsules grâce à une différence de pression de chaque côté de la microcapsule. Ce système est déjà présent dans les hôpitaux et dans les grandes surfaces.
\begin{figure}[h!]
    \centering
    \begin{subfigure}{0.45\textwidth}
        \centering
        \includegraphics[width=\linewidth]{assets/figures/Hardware/transport_pneu/reseau_pneumatique_hopital.jpg}
        \caption{Schéma d'un réseau de transport pneumatique\footnotemark}
    \end{subfigure}\hfill
    \begin{subfigure}{0.45\textwidth}
        \centering
        \includegraphics[width=\linewidth]{assets/figures/Hardware/transport_pneu/cartouche_transport_pneu.jpg}
        \caption{Cartouche de transport\footnotemark}
    \end{subfigure}
    \caption{Exemples de réseau de transport pneumatique par tube}
\end{figure}
\footnotetext[1]{\href{https://www.transport-pneumatique.fr/transport-pneumatique-centres-hospitaliers/}{https://www.transport-pneumatique.fr/transport-pneumatique-centres-hospitaliers/}}
\footnotetext[2]{\href{https://www.transport-pneumatique.fr/cartouches-pochettes/}{https://www.transport-pneumatique.fr/cartouches-pochettes/}}
\subsection*{Transport par convoyeur}
Pour déplacer les microcapsules, un convoyeur peut être utilisé, il faut néanmoins que le convoyeur soit adapté au microcapsule, les microcapsules étant cyclindriques, elles risquerait de rouler sur un convoyeur à bande lisse, mais une bande à tasseau (\cf \autoref{img:convoyeur_bande_tasseau}) ou un demi-tube  (\cf \autoref{img:convoyeur_tube}) conviendraient parfaitement.
\begin{figure}[h!]
    \centering
    \begin{subfigure}{0.45\textwidth}
        \centering
        \includegraphics[width=\linewidth]{assets/figures/Hardware/transport_conv/convoyeur_tasseau.JPG}
        \caption{Convoyeur avec bande à tasseaux\footnotemark}
        \label{img:convoyeur_bande_tasseau}
    \end{subfigure}\hfill
    \begin{subfigure}{0.45\textwidth}
        \centering
        \includegraphics[width=\linewidth]{assets/figures/Hardware/transport_conv/convoyeur_tube.png}
        \caption{Convoyeur à tube\footnotemark}
        \label{img:convoyeur_tube}
    \end{subfigure}
    \caption{Exemple de convoyeur}
\end{figure}
\footnotetext[2]{\href{https://fr.m.wikipedia.org/wiki/Convoyeur}{https://fr.m.wikipedia.org/wiki/Convoyeur}}
\footnotetext[3]{\href{https://doser-compter.com/products/ligne-de-comptage-king}{https://doser-compter.com/products/ligne-de-comptage-king}}
Quant aux différents moyens de mouvoir les microcapsules, il y a : 
\begin{itemize}
    \item les vibrations;
    \item le déplacement de la bande;
    \item la gravité.
\end{itemize}

La dernière option nécessite des surface lisse, que le système soit en pente et le temps de déplacement n'est pas réglable. Les deux autres solutions ne se distinguent pas vraiment pour l'instant, car dans tous les cas, l'utilisation d'un moteur électrique est nécessaire.

\subsection*{Déplacement à l'aide d'un robot}
Pour le déplacement des microcapsules, seuls les axes $T_x, T_y~\text{et}~T_z$ sont nécessaires, soit $3$ degrés de liberté. Un robot de type \textit{SCARA}, cylindrique ou Delta peuvent correspondre.

\subsection*{Avantages et inconvénients}
\begin{table}[H]
    \caption{Anvantages et inconvénients des solution de transport des microcapsules}
    \begin{tabular}{@{}cll@{}}
    \toprule
    Solution      & \multicolumn{1}{c}{Avatanges}                                                                                                   & \multicolumn{1}{c}{Inconvénient}                                                                                                                                  \\ \midrule
    Transport pneumatique par tube    & \begin{tabular}[c]{@{}l@{}}\end{tabular} & \begin{tabular}[c]{@{}l@{}}- Peu modulable\\ - Bruyant\\ - Aiguillage complexe\end{tabular}                                                                     \\
    Convoyeur & \begin{tabular}[c]{@{}l@{}}- \\ - \end{tabular} & \begin{tabular}[c]{@{}l@{}}- Maintenance fréquente\\ - Ne convient pas au petits objets\\ - Espace limité, pour pouvoir ouvrir \\ et fermer la pince \\ - Nécessite un contrôle de force\end{tabular} \\
    Robot   & \begin{tabular}[c]{@{}l@{}}- Place\\ - Modulable \\ \end{tabular}                            & \begin{tabular}[c]{@{}l@{}}- Coût\\ - Nécessite un nettoyage pour conserver \\  l'adhérence dans le temps\\ - Détachement complexe\end{tabular}     \\ \bottomrule
    \end{tabular}
\end{table}

\subsection{Analyse des solutions}
De par la complexité et le manque de modularité du transport pneumatique et du convoyeur, le choix d'utiliser un robot a été choisi.

La position des plaques dans la \textit{glove box} est importante pour la suite, elle permettra de choisir le type de robot à utilisés.
Il est important de savoir comment placer les deux plaques dans la \textit{glove box}.
\subsubsection{Positionnement des plaques dans la \textit{glove box}}
Pour la position des plaques, les solutions trouvées sont : 
\begin{enumerate}
    \item Les plaques ne bougent pas et restent dans les sas;
    \item Les plaques sont positionées symétriquement par rapport au centre de la \textit{glove box};
    \item La plaque de microcapsules est déplacée à côté de la plaque de réacteur, cette dernière ne bouge pas;
    \item La plaque de microcapsules reste dans le sas tandis que la plaque de réacteur est transportée à ses côtés;
    \item Les deux plaques sont mises l'une à côté de l'autre en étant plus proches du stock;
    \item Les deux plaques sont mises l'une à côté de l'autre en étant plus proches de la sortie;
\end{enumerate}
Le temps de cycle est le suivant :
\begin{align*}
    tpsCycle &= tpsDeplacementPlaquesCapsule + tpsDeplacement_{Capsule} \\
    &+ tpsAttentePlaque + tpsDeplacement_{PlaquesReacteur}\\
    &= nbrePlaques\cdot \frac{d_{PlaqueMicrocapsules}}{v_{robot}} + nbreMicrocapsules\cdot \frac{d_{EntrePlaque}}{v_{robot}} \\
    &+ 2\cdot nbrePlaques\cdot tpsAttente + \frac{d_{PlaqueReacteur}}{v_{robot}}
\end{align*}

Certaines valeurs ne peuvent être connue que lors de la mise en route, notamment l'accélération, la vitesse d'approche, il est nécessaire de faire certaines hypothèses. Les distances sont également arbitraires afin de se faire une idée du temps de cycle des solutions. Ici, la position optimale n'est pas recherché.
En utilisant les hypothèses suivantes :
\begin{itemize}
    \item $v_{robot} = \qty{1}{\frac{\m}{\s}}$;
    \item $d_{EntrePlaque} = \qtylist[list-units = single]{1.33; 0.3; 0.3; 0.3; 0.3; 0.3}{\m\per\s}$;
    \item $d_{PlaqueMicrocapsules} = \qtylist[list-units = single]{0; 0.515; 1.03; 0; 0.83; 0.2 }{\m}$;
    \item $d_{PlaqueReacteur} =      \qtylist[list-units = single]{0; 0.515; 0; 1.03; 0.2; 0.83 }{\m}$;
    \item $d_{PlaqueMicrocapsules} = \qtylist[list-units = single]{0; 0.515; 1.03; 0; 0.83; 0.2 }{\m}$;
    \item L'accélération du robot est infinie;
    \item La vitesse d'approche n'est pas prise en compte.
\end{itemize}
Avec ces informations, il est possible de calculer le temps moyen d'un cycle (\cf \autoref{img:graph_temps_cycle_moyen_sequentielle}) en fonction du nombre de microcapsule demandées ainsi que du nombre de microcapsule par plaque, et ce, pour chaque solution.
\begin{figure}[H]
    \centering
    \includesvg[width = \textwidth]{assets/figures/Hardware/recherchSoluce/temps_cycle_moyen_toute_solution.svg}
    \caption{Temps de cycle moyen des différentes solutions}
    \label{img:graph_temps_cycle_moyen_sequentielle}
\end{figure}
Il est possible de voir que les solutions $1$ et $6$ semble les plus rapide. Le temps d'attente des plaques est cependant très important, environ $84~\%$ du temps total. Pour réduire ce délai, il peut être intéressant de paralléliser les tâches.
\subsubsection{parallélisation des robots}
La parallélisation des tâches consiste à effectuer le \textit{pick and place} des microcapsules, la prise et la dépose des plaques indépendamment.
De par la configuration des sas, il n'est possible d'y mettre qu'une seule plaque, la parallélisation des solutions $\numlist{1; 3; 4}$ n'est donc pas possible.

\begin{figure}[H]
    \centering
    \includesvg[width = \textwidth]{assets/figures/Hardware/recherchSoluce/para.svg}
    \caption{Comparaison du temps de cycle des solutions en fonction du nombre de microcapsules par plaque}
    \label{img:graph_temps_cycle_para}
\end{figure}

Sur \autoref{img:graph_temps_cycle_para}, il est possible de voir que les solutions parallélisées ont toujours un temps de cycle inférieur aux solutions séquentielles tant que le nombre de microcapsules par plaque est inférieur au nombre de microcapsules demandées.
Plus le nombre de microcapsule demandées est élevé, plus le gain de temps est significatif, allant, en moyenne, de $1.3~\%$ pour $48$ microcapsules à $27~\%$ pour $240$ microcapsules.
\section{Présentation de la solution choisie}
L'utilisation d'un seul robot positionné au centre de la \textit{glove box}, qui s'occupera de déplacer les plaques ainsi que de faire le \textit{pick and place} des microcapsules a été retenue.
Cette décision est due à : 
\begin{itemize}
    \item un robot déjà présent dans le laboratoire;
    \item la grande flexibilité de cette solution (la \textit{glove box} sera utilisée pour d'autres processus par la suite).
\end{itemize}
\section{Description du matériel utilisé}
Voici la liste de matériel utilisé lors du TB : 
\begin{itemize}
    \item Robot \og{}UR3e\fg{};
    \item Pince \og{}Hand-E\fg{};
    \item Adaptation de la pince précédente sur mesure;
    \item Module d'aspiration;
    \item Support pour les plaques.
\end{itemize}
\section{Intégration}
Afin de maximiser l'espace accessible par le robot, il a été surélevé (\cf \autoref{img:integration_robot_glove}). La moitié gauche est réservée pour le \textit{pick and place} des microcapsules, tandis que la partie à droite servira pour la suite du processus.
Pour les outils, la pince se met sur la bride du robot et elle possède deux trous taraudés qui seront utilisé pour fixer le module d'aspiration (\cf \autoref{img:integration_pince}).
L'outil pour saisir les plaques fait un angle à $\qty{45}{\degree}$ pour pouvoir déposer les plaques sur tout le plan \textit{repPickAndPlace}. Afin de gagner de la place, les deux outils sont placés dans des directions opposées.
\begin{figure}[]
    \centering
    \includegraphics[width = 0.5\textwidth]{assets/figures/Hardware/gloveBox.jpeg}
    \caption{Intégration du robot dans la \textit{glove box}}
    \label{img:integration_robot_glove}
\end{figure}
\begin{figure}[]
    \centering
    \includegraphics[width = 0.5\textwidth]{assets/figures/Hardware/outil_complet.jpeg}
    \caption{Module d'aspiration}
    \label{img:integration_pince}
\end{figure}
\chapter{Software}
\section{Algorithme de recombinaison}
\subsection{Spécification de l'algorithme}
\subsubsection*{Objectif}
L'algorithme a pour objectif de sélectionner les microcapsules, dans le stock, à utiliser pour chaque recette d'un batch, soit environ $300$ recettes.

\subsection*{Contraintes}
Les contraintes de l'algorithme sont les suivantes : 
\begin{enumerate}
    \item Maximiser le nombre de recettes réalisées.
    \begin{itemize}
        \item Respecter les quantités délivrées;
        \item respecter la quantité maximale de microcapsules de chaque réacteur.
    \end{itemize}    
    \item Minimiser le nombre de microcapsules utilisées par réacteur.
\end{enumerate}
\subsection*{Entrées}
Il possède en entrée : 
\begin{itemize}
    \item les recettes du batch;
    \item la position et la quantité de produit des microcapsules disponibles dans le stock;
    \item la tolérance des recettes;
    \item le nombre maximal de microcapsule par réacteur.
\end{itemize}

\subsection*{Sorties}
Les sorties de l'algorithme incluent : 
\begin{itemize}
    \item les numéros de plaque à utiliser;
    \item la position des microcapsules sur les plaques (sous la forme \og{}ligne, colonne\fg{} );
    \item les coordonnées et les numéros d'identification des réacteurs dans lesquels les microcapsules doivent être placées;
    \item les recettes non réalisables, avec indication des éléments manquants;
    \item les recettes réalisées.
\end{itemize}
\subsection{Méthode d'optimisation et stratégie d'ordonnancement}
% \section{Robot}
% La programmation du robot est faite en deux parties, le robot contient les programmes de déplacement 
% tandis qu'un programme python s'occupera d'appeler les programmes et de transmettre les informations nécessaire au robot.
% Les paramères seront données au différents programmes grâce aux registres du robot.
% \subsection{Programmation du robot}
% \subsubsection{Outils}
% Le robot possède $2$ outils, une pince et un module d'aspiration, ces outils sont sur la même platforme, il n'est donc pas nécessaire de faire de changement d'outils, cependant, il est important de configurer les deux outils indépendanment dans le robot car leur TCP n'est pas au même endroit.
% \subsubsection{Programmes}
% Pour la recombinaison, le robot nécessite $3$ programmes différents :
% \begin{itemize}
%     \item PickAndPlaceVial(int lineVial, int columnVial, int lineReactor, int coumnReactor), qui permet de faire le déplacement des microcapsule de leur plaque jusqu'au réacteur;
%     \item TakePlate(bool isParadox), permettant de récupérer un plaque de réacteur ou de microcapsules;
%     \item GivePlate(bool isParadox), qui rendra les plaques qui ne sont plus nécessaire dans la glove box.
% \end{itemize}
% \subsubsection{Plans}
% \subsubsection{Sécurité}
% Afin que le robot ne rentre pas en colision avec les portes de la \textit{glove box}, il est nécessaire d'installer des limites, ces limites sont un plan que le TCP ne peut pas franchir.
% Chaque défaut sera écrit dans un registre afin de pouvoir transmettre les erreurs au programme python.
% \subsection{Communication avec le robot}
% Afin de communiquer avec le robot, l'utilisation de l'interface \textit{Real-Time Data Exchange} (RTDE), présent sur les robots d'Universal robot (UR), qui permet l'échange de données bidirectionnel en temps réel, entre le robot et le système externe. 
% L'utilisation de l'API \og{} dashboard\_client permet \fg{} de charger et de démarrer des programmes robot présent dans la mémoire ce dernier. Tandis que \og{} RTDIOInterface \fg{} permet d'écrire et de lire les entrées, sorties et registres du robot.
\section{Robot}

\subsection{Objectif}
Le robot doit manipuler les microcapsules en effectuant des tâches de \textit{pick-and-place} depuis leur zone de stockage jusqu’aux réacteurs, en fonction des besoins de l'algorithme de recombinaison.

\subsection{Contraintes}
\begin{enumerate}
    \item Respect des limites de la \textit{glove box} pour éviter toute collision;
    \item Gestion de deux outils (pince et module d'aspiration) sans changement manuel;
\end{enumerate}

\subsection{Entrées}
Les informations nécessaires pour le contrôle du robot incluent :
\begin{itemize}
    \item les coordonnées des microcapsules et des réacteurs (ligne et colonne sur les plaques) ;
\end{itemize}

\subsection{Sorties}
Les sorties du programme de contrôle du robot sont :
\begin{itemize}
    \item les informations d'état du robot et les erreurs enregistrées ;
\end{itemize}

\subsection{Méthode de programmation}
La programmation est effectuée en deux parties :
\begin{itemize}
    \item Les programmes intégrés au robot gèrent les déplacements : \og{}PickAndPlaceVial\fg{}, \og{}TakePlate\fg{}, et \og{}GivePlate\fg{} ;
    \item Un programme externe en Python appelle les fonctions du robot et transmet les informations nécessaires via l'interface RTDE (cf. \autoref{subsection:CommunicationRobot}).
\end{itemize}

\subsection{Sécurité}
Des plans de sécurité sont configurés pour que le robot n'entre pas en collision avec les portes de la \textit{glove box}. Les erreurs sont enregistrées dans un registre pour être relayées au programme Python.

\subsection{Communication avec le robot}\label{subsection:CommunicationRobot}
L'interface \textit{Real-Time Data Exchange} (RTDE) permet l'échange bidirectionnel en temps réel des données entre le robot et le système externe, facilitant ainsi le démarrage des programmes robot et la lecture/écriture des registres nécessaires.

\section{Interaction entre le logiciel et le matériel pour la recombinaison}


\chapter{Implémentation et intégration Hardware-Software}
\section{Développement du logiciel}
\subsection{Objets}
Afin de faciliter l'améliorabilité et la clareté du code, une programmation orientée objet à été choisi (Diagramme de classe \autoref{fig:diag_classe})
\begin{figure}[ht]
    \centering
    \includegraphics[width=\linewidth]{assets/figures/Diagramme_classe.drawio}
    \caption{Diagramme de classe}
    \label{fig:diag_classe}
\end{figure}
\subsection{Optimisateur}
Pour l'optimisateur, celui-ci suit le diagramme de flux (cf. \autoref{fig:algorithme_solver}) avec a la suite une traitement des données récoltées.
\begin{figure}
    \centering
    \includegraphics[width=0.8\linewidth]{assets/figures/Diagramme_flux_opt.drawio}
    \caption{Diagramme de flux de l'optimisateur implémenté}
    \label{fig:algorithme_solver}
\end{figure}
L'utilisation de SCIP\footcite{SCIP} (qui permettra de ne pas implémenter chaque algorithme indépendamment) à travers le wrapper or-tools\footcite{ORTOOLS} sera utilisé afin de faciliter l'intégration en python.

\section{Tests de validation et calibrage du matériel}
\subsection{Tests unitaires}
Les tests unitaires des classes et de l'optimisateur se font avec les modules \og pytest \fg (\textcolor{red}{Ajout de référence code}) et \og \textit{Coverage} \fg. Ce dernier permet de connaître le pourcentage de lignes de code tester ($94\%$ dans ce cas (cf. \autoref{fig:coverage_global})).
\begin{figure}[H]
    \centering
    \includegraphics[width=1\linewidth]{assets/figures/coverage.png}
    \caption{Couverture du code par les tests unitaires}
    \label{fig:coverage_global}
\end{figure}
Pour le robot, les tests se sont fait en lançant les programmes et en vérifiant visuellement que le comportement soit correct. 

\chapter{Sécurité et Préventions des Risques}
La sécurité est un aspect important de ce projet, en raison de l'utilisation d'un robot UR$3$ dans un espace confiné, sous azote, pour manipuler des composants chimiques qui ne doivent pas entrer en contact, car ces éléments peuvent être réactifs entre eux. Cette section détaille les mesures mises en \oe{}uvre ou à mettre en \oe{}uvre pour assurer la sécurité des opérateurs, du matériel et des échantillons.
\section{Identification des risques}\label{sec:identification_risque}
Pour faire cette analyse, une approche ascendante\footnote{L'approche ascendante consiste à partir de la base (qu'est-ce qui peut être dangereux) puis remonter pour en trouver les effets. Inversement à l'approche descendante qui cherche les causes à partir d'un accident.} avec une matrice de risques (cf. \autoref{tab:ananylse_risque}) est utilisées. Pour la matrice de risque, la criticité du risque est le produit de la gravité et de l'occurrence de ce risque. Cette criticité servira à prioriser l'ordre de réduction des risques (plus la criticité est élevée, plus le risque est important et doit être réduit).
\rowcolors{3}{gray!10}{white}
\begin{landscape}
    \begin{table}[ht]
        \begin{tabular}{l|l|l|c|c|c}
        \multicolumn{1}{c|}{Id} & \multicolumn{1}{|c|}{Description}              & \multicolumn{1}{c|}{Impact}                                                                                                                    & \begin{tabular}[c]{@{}c@{}}Gravité\\ {[}1-5{]}\end{tabular} & \begin{tabular}[c]{@{}c@{}}Occurence\\ {[}1-5{]}\end{tabular} & \begin{tabular}[c]{@{}c@{}}Criticité\\ {[}1-25{]}\end{tabular} \\
        \hline
        1                      & Déchirement des gants par le robot              & Perte de l'étanchéité                                                                                                                          & 5                                                           & 1                                                             & \cellcolor{yellow} 5                                                              \\
        2                      & \gls{microcapsule} non saisie             & Erreur lors des analyses                                                                                                                       & 1                                                           & 2                                                             & \cellcolor{green}  2                                                              \\
        3                      & \gls{microcapsule} non déposée            & \begin{tabular}[c]{@{}l@{}}Erreur lors des analyses\\ Détérioration des prochaines \glspl{microcapsule}\end{tabular}                            & 3                                                           & 2                                                             & \cellcolor{yellow} 6                                                              \\
        4                      & Perte de \gls{microcapsule} lors du déplacement & \begin{tabular}[c]{@{}l@{}}Erreur lors des analyses\\ Contamination de la \gls{glovebox}\end{tabular}                                          & 4                                                           & 3                                                             & \cellcolor{orange} 12                                                             \\
        5                      & Renversement d'une \gls{wellplate}                    & \begin{tabular}[c]{@{}l@{}}Perte de la totalité des capsules de la \gls{wellplate} \\ Contamination de la \gls{glovebox}\end{tabular}                 & 5                                                           & 2                                                             & \cellcolor{orange} 10                                                             \\
        6                      & Renversement d'une \gls{paradox}                & \begin{tabular}[c]{@{}l@{}}Perte de la totalité des capsules présentes dans la \gls{paradox}\\ Contamination de la \gls{glovebox}\end{tabular} & 5                                                           & 3                                                             & \cellcolor{red}    15                                                             \\
        7                      & Collision entre les gants et le robot           & Contusions                                                                                                                                     & 5                                                           & 1                                                             & \cellcolor{yellow} 5                                                             
        \end{tabular}
        \caption{Analyses des risques}
        \label{tab:ananylse_risque}
        \end{table}
\end{landscape}

\section{Solutions}
Cette section à
a pour but de donner des solutions envisageables aux risques déterminés précédemment (cf. \autoref{sec:identification_risque}). Les solutions trouvées doivent impacter la gravité, le nombre d'occurrences ou les deux.
Par exemple, en prenant le risque à traiter en priorité (le numéro $6$ qui possède une criticité de $15$ (cf. \autoref{tab:ananylse_risque})), il est possible de modifier la \gls{paradox} afin d'y ajouter des ergots et ainsi améliorer le maintient de celles-ci dans la pince du robot, réduisant le nombre d'occurrences de ce risque. Mais il est également possible de rajouter un couvercle par-dessus la \gls{paradox} afin de ne pas éparpiller le contenu dans la \gls{glovebox}, réduisant ainsi la gravité du risque.
Toutes les solutions proposées n'ont pas forcément été appliquées lors du travail de Bachelor.

\begin{table}[H]
    \begin{tabular}{l|c|c}
        Id du risque & Solutions proposées                                                                                                                                          &\begin{tabular}[c]{@{}c@{}}Action{[}Gravité,\\Occurence ou les deux{]}\end{tabular}\\
        \hline
        $6$          & \begin{tabular}[c]{@{}l@{}}Améliorer l'ergonomie de la \gls{paradox} par rapport à la pince.\\Placer un couvercle sur la \gls{paradox} lors des déplacements.\end{tabular} & \begin{tabular}[c]{@{}l@{}}Occurence\\Gravité\end{tabular}\\
        $4$          & \begin{tabular}[c]{@{}l@{}}Augmenter la pression pour l'aspiration.\\ Mettre un bac récupérateur sous la pince.\end{tabular}                                   & \begin{tabular}[c]{@{}l@{}} Occurence\\ Gravité\end{tabular}\\
        $5$          & \begin{tabular}[c]{@{}l@{}}Ajout de trous sur la \gls{wellplate}\\pour améliorer la prise de la pince.\end{tabular}                                        & Occurence\\
        $3$          & Ajouter un système de soufflerie lors de la dépose.                                                                                                            & Occurence\\
        $7$          & Mettre des gants plus épais.                                                                                                                                   & Gravité\\
        $1$          & Arrondir tous les angles présents sur la pince.                                                                                                                & Occurence\\
        $2$          & Augmenter la pression pour l'aspiration.                                                                                                                       & Occurence
    \end{tabular}
    \caption{Solutions face aux risques (dans l'ordre de priorité)}
    \label{tab:solutions_risques}
\end{table}
\chapter{Résultat et Analyse}
\section{Performance du système}
\section{Analyse des erreurs}

\chapter{Discussion}
\section{Points forts}
La réalisation de ce travail de Bachelor a permis de concevoir et d'implémenter un système robotisé pour la manipulation de \glspl{microcapsule} dans un environnement confiné. Bien que les objectifs initiaux aient été globalement atteints, certaines limitations et axes d'amélioration méritent d'être discutés.
\begin{itemize}
    \item \textbf{Précision et efficacité :} Le système robotisé mis en place s'est révélé capable de manipuler les \glspl{wellplate} avec une précision satisfaisante. L'intégration du robot avec les outils dévellopés a permis d'automatiser efficacement le transfert des capsules entre les \glspl{wellplate} et les réacteurs.
    \item \textbf{Flexibilité du système :} La configuration retenue, avec un robot positionné au centre de la \glspl{glovebox}, offre une grande modularité, permettant de l'adapter à d'autres processus futures au laboratoire SwissCar$+$.
    \item \textbf{Optimisation logicielle :} L'algorithme de recombinaison a démontré sa capacité à maximiser le nombre de recettes réalisables, grâce à une approche rigoureuse basée sur des techniques d'optimisation contraintes.
\end{itemize}
\section{Limitations}
\begin{itemize}
    \item \textbf{Prise des microcapsules :} Le système pneumatique pour saisir les \glspl{microcapsule} n'as pas pu être dévellopé par manque de temps.
    \item \textbf{Problème de manipulation :} Certains risques liés à la manipulation, comme le renversemment des \glspl{wellplate}, des \glspl{paradox} ou la perte de \glspl{microcapsule} lors du transport, ont été identifiés. Des solutions ont été proposées (cf. \autoref{tab:solutions_risques}), leur implementation reste partielle voir inexistante et nécessiterait une validation plus approfondie.
    \item \textbf{Temps de traitement : } Bien que l'algorithme permette une recombinaison efficace des \glspl{microcapsule}, lors de problèmes trop complexe, la recombinaison pourrait devenir un goulot d'étranglement. Une optimisation supplémentaire de l'algorithme, notamment en utilisant une méthode heuristique plus avancée pourrait améliorer les performances. 
\end{itemize}
\section{Perspectives}
\begin{itemize}
    \item \textbf{Amélioration matérielle :} L'amélioration de la pince pour pouvoir saisir la \gls{paradox} (un entraxe plus grand suffirait).
    \item \textbf{Dévellopement du réseau pneumatique :} Le dévellopement d'un réseau pneumatique permettant au robot de faire le plus de déplacement possible est nécessaire.
    \item \textbf{Validation expérimentale :} Des tests à plus grandes échelle avec des lot de \glspl{microcapsule} et des recettes variées permettront d'évaluer plus précisement les performances du systèmes en conditions réelles.
    \item \textbf{Algorithme et intelligence artificielle :} L'intégration d'un algorithme d'apprentissage pour sélectionner les \glspl{microcapsule} permettrait de diminuer le temps de traitement de l'algorithme.
\end{itemize}

\chapter{Conclusion}
Ce projet de Bachelor a permis de dévelloper un système robotisé innovant pour la recombinaison des \glspl{microcapsule} chimiques dans un laboratoire de recherche. Les contributions principales incluent la conception et l'integration d'un robot dans une \gls{glovebox}, et la création d'un algorithme d'optimisation permettant de maximiser le nombre de recette réalisable.

L'ensemble du système répond aux éxigences initiales initiales en matières de précision, de fiabilité et de sécurité, tout en offrant une grande flexibilité pour des applications futures.Les résultats obtenus démontrent que l'automatisation de la recombinaison peut considérablement accélérer le processus expérimental, réduisant les erreurs humaines et augmentant l'efficacité global.

Cependant certains points, comme le réseau pneumatique, n'ont pas pu être réalisée à cause d'un manque de temps.

Des axes d'amélioration existent. Le temps de calcul de l'algorithme, la robustesse des outils et la gestion des risques liés à la manipulation sont des aspects à améliorer. L'ajout d'un module d'intelligence artificielle et l'amélioration des outils physique pourrait augmenter le rendement de la recombinaison.

En conclusion, ce travail de représente une avancée vers l'automatisation complète des tâches dans le laboratoire.

\vfil
\hspace{8cm}\makeatletter\@author\makeatother\par
\hspace{8cm}\begin{minipage}{5cm}
\end{minipage}


\clearpage
\printbibliography

\appendix
\appendixpage
\addappheadtotoc
\chapter{Optimisateur}\label{ann:optimisateur_annexe}
\begin{listing}[H]
    \lstinputlisting[firstline=1, lastline=32]{assets/code/solver.py}
    \caption{Génération d'un diagramme de Bode \label{python}}
\end{listing}
\begin{listing}[H]
    \lstinputlisting[firstline=33, lastline=76]{assets/code/solver.py}
    \caption{Génération d'un diagramme de Bode \label{a}}
\end{listing}
\begin{listing}[H]
    \lstinputlisting[firstline=77, lastline=124]{assets/code/solver.py}
    \caption{Génération d'un diagramme de Bode \label{b}}
\end{listing}
\begin{listing}[H]
    \lstinputlisting[firstline=125, lastline=172]{assets/code/solver.py}
    \caption{Génération d'un diagramme de Bode \label{c}}
\end{listing}
\begin{listing}[H]
    \lstinputlisting[firstline=172, lastline=222]{assets/code/solver.py}
    \caption{Génération d'un diagramme de Bode \label{d}}
\end{listing}
\begin{listing}[H]
    \lstinputlisting[firstline=222, lastline=273]{assets/code/solver.py}
    \caption{Génération d'un diagramme de Bode \label{e}}
\end{listing}
\begin{listing}[H]
    \lstinputlisting[firstline=274, lastline=320]{assets/code/solver.py}
    \caption{Génération d'un diagramme de Bode \label{f}}
\end{listing}
\begin{listing}[H]
    \lstinputlisting[firstline=321, lastline=344]{assets/code/solver.py}
    \caption{Génération d'un diagramme de Bode \label{g}}
\end{listing}

\chapter{Tests unitaires}\label{ann:test_unitaire}
\begin{listing}[H]
    \lstinputlisting[firstline=1, lastline=42]{assets/code/test_batch.py}
    \caption{Test unitaire de batch.py \label{code:test_batch_unitaire}}
\end{listing}
\begin{listing}[H]
    \lstinputlisting[firstline=43, lastline=49]{assets/code/test_batch.py}
    \caption{Test unitaire de batch.py \label{code:test_batch_unitaire_2}}
\end{listing}
\begin{listing}[H]
    \lstinputlisting{assets/code/test_recipe.py}
    \caption{Test unitaire de recipe.py \label{code:test_recipe_unitaire}}
\end{listing}
\begin{listing}[H]
    \lstinputlisting[firstline=1, lastline=60]{assets/code/test_solve.py}
    \caption{Test unitaire de solve.py \label{code:test_solve_unitaire1}}
\end{listing}
\begin{listing}[H]
    \lstinputlisting[firstline=61, lastline=116]{assets/code/test_solve.py}
    \caption{Test unitaire de solve.py \label{code:test_solve_unitaire2}}
\end{listing}
\begin{listing}[H]
    \lstinputlisting[firstline=117, lastline=172]{assets/code/test_solve.py}
    \caption{Test unitaire de solve.py \label{code:test_solve_unitaire3}}
\end{listing}
\begin{listing}[H]
    \lstinputlisting[firstline=172, lastline=227]{assets/code/test_solve.py}
    \caption{Test unitaire de solve.py \label{code:test_solve_unitaire4}}
\end{listing}
\begin{listing}[H]
    \lstinputlisting[firstline=227, lastline=243]{assets/code/test_solve.py}
    \caption{Test unitaire de solve.py \label{code:test_solve_unitaire5}}
\end{listing}
\begin{listing}[H]
    \lstinputlisting[firstline=1, lastline=57]{assets/code/test_storage.py}
    \caption{Test unitaire de storage.py \label{code:test_storage_unitaire1}}
\end{listing}
\begin{listing}[H]
    \lstinputlisting[firstline=58, lastline=100]{assets/code/test_storage.py}
    \caption{Test unitaire de storage.py \label{code:test_storage_unitaire2}}
\end{listing}
\begin{listing}[H]
    \lstinputlisting{assets/code/test_vial.py}
    \caption{Test unitaire de vial.py \label{code:test_vial_unitaire}}
\end{listing}

\begin{sidewaystable}[ht]
    \begin{tabular}{@{}c|p{3cm}|p{3cm}|p{3cm}|p{3cm}|p{3cm}|p{3cm}|c@{}}
    \toprule
    N$^\circ$ de test          & Fonction                            & Description                                                                                                                                                                                & Entrées            & \multicolumn{1}{c}{Tests effectués}           & Résultat attentdu                                                                                & \multicolumn{1}{c}{Résultat}                                                                     & Validation                \\ \midrule
    \multirow{2}{*}{1}  & \multirow{2}{*}{connection()}       &  \multirow{2}{*}{Test de la fonction de "connection" avec une ip correct et une incorrect. Le résultat est attendue sur le terminal de commande}                                           & ip : "192.168.1.2  & - Tentative de connexion avec une bonne ip    & " Connection with the robot (ip : 192.168.1.2) is etablished"                                    & Connection with the robot (ip : 192.168.1.2) is etablished                                       & \cellcolor[HTML]{B2FFB2} \\
                        &                                     &                                                                                                                                                                                            & ip : "192.168.1.3" & - Tentative de connexion avec une mauvaise ip & "Connection error : "Timeout connecting to UR dashboard server."                                 & "Connection error : "Timeout connecting to UR dashboard server."                                 & \cellcolor[HTML]{B2FFB2} \\
    2                   & pick\_wellplate()                   & Test du fonctionnement normal de la fonction pick\_wellplate                                                                                                                               &                    & - Appel de la fonction avec URRTDE            & Prise de la wellplate dans le sas et dépose à la place dans la glove box                         & Prise de la wellplate dans le sas et dépose à la place dans la glove box                         & \cellcolor[HTML]{B2FFB2} \\
    3                   & give\_wellplate()                   & Test du fonctionnement normal de la fonction give\_wellplate                                                                                                                               &                    & - Appel de la fonction avec URRTDE            & Prise de la wellplate dans la glove box puis dépose dans le sas                                  & Prise de la wellplate dans la glove box puis dépose dans le sas                                  & \cellcolor[HTML]{B2FFB2} \\
    4                   & pick\_and\_place\_vials()           & Test du fonctionnement normal de la fonction pick\_and\_place\_vials                                                                                                                       & ("P24", "C3")      & - Appel de la fonction avec URRTDE            & Prise de la capsule situer en P24 sur la wellplate puis dépose de celle-ci en C3 dans la paradox & Prise de la capsule situer en P24 sur la wellplate puis dépose de celle-ci en C3 dans la paradox & \cellcolor[HTML]{FFB2B2} \\
    5                   & \multicolumn{1}{l}{pick\_paradox()} & Test du fonctionnement normal de la fonction pick\_paradox                                                                                                                                 &                    & - Appel de la fonction avec URRTDE            & Prise de la paradox dans le sas et dépose à la place dans la glove box                           &                                                                                                  & \cellcolor[HTML]{FFB2B2} \\
    6                   & \multicolumn{1}{l}{give\_paradox()} & Test du fonctionnement normal de la fonction give\_paradox                                                                                                                                 &                    & - Appel de la fonction avec URRTDE            & Prise de la paradox dans la glove box et dépose dans le sas                                      &                                                                                                  & \cellcolor[HTML]{FFB2B2} \\ \bottomrule
    \end{tabular}
\end{sidewaystable}
\let\cleardoublepage\clearpage
\backmatter

\label{glossaire}
\printnoidxglossary
\label{index}
\printindex

% Le colophon est le dernier élément d'un document qui contient des notes de l'auteur concernant la mise en page et l'édition du document : il est parfaitement optionnel.
\input{colophon.tex}

\end{document}
