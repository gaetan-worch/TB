\section{Problématique}
Il est nécessaire de s'assurer de l'optimisation du nombre de \glspl{recette} réalisées dans un \gls{batch} afin d'obtenir des résultats fiables. Cependant, le nombre du combinaison de \glspl{microcapsule}, possible pour un \gls{batch} devient rapidement trop important pour pouvoir être réalisées par la méthode de \gls{bruteforce}.
\section{Objectifs}
Concevoir et développer une solution permettant le transfert automatisé, précis et rapide de \glspl{microcapsule} de réactif chimique entre le stockage et les réacteurs. Un algorithme devra être conçu, qui, en comparant le stock avec le \gls{batch}, optimisera les microcapsules à sélectionner pour chaque réacteur et devra avertir si des \glspl{recette} ne sont pas réalisables.
%  Un algorithme devra être conçu pour optimiser la sélection des microcapsules afin de maximiser le nombre de \glspl{recette} réalisées, en comparant le stock avec le \glspl{batch}. L'ensemble du système devra garantir un haut niveau de fiabilité et de sécurité dans le processus.
\section{Cahier des charges fonctionnel}
\rowcolors{3}{gray!10}{white}
\begin{longtable}{l|m{5cm}|m{5cm}}
    \caption{Cahier des charges fonctionnel}\\
    \hline \multicolumn{1}{c|}{\textbf{fonction}} & \multicolumn{1}{c|}{\textbf{énoncé de la fonction}} & \multicolumn{1}{c}{\textbf{éxigence}} \\ \hline 
    \endfirsthead
    
    \multicolumn{3}{c}%
    {{\textbf{\tablename\ \thetable{}}-- continued from previous page}} \\
    \hline \multicolumn{1}{c|}{\textbf{fonction}} & \multicolumn{1}{c|}{\textbf{énoncé de la fonction}} & \multicolumn{1}{c}{\textbf{éxigence}} \\ \hline 
    \endhead
    \hline \multicolumn{3}{r}{{Continued on next page}} \\ \hline
    \endfoot
    \hline \hline
    \endlastfoot
    FP $1$&\centering Manipuler des microcapsules de manière automatisée, sans les endommagées&\begin{itemize}
            \item Ne pas détériorer la microcapsule
        \end{itemize}\\
        FP $1.1$&\centering Prélever les microcapsules dans une plaque& \begin{itemize}
            \item Position de prise arbitraire
            \item Contrôler que la microcapsule soit saisie
        \end{itemize}\\
        FP $1.2$&\centering Déplacer les microcapsules&\\
        FP $1.3$&\centering Déposer les microcapsules&\begin{itemize}
            \item Position de dépose arbitraire dans la plaque de réacteurs
        \end{itemize}\\
        FP $2$&\centering Déterminer les microcapsules les plus adaptées pour chaque réacteur, selon une recette donnée&\\
        FP $2.1$&\centering Recevoir la recette pour chaque réacteur&\\
        FP $2.2$&\centering Accéder à la base de donnée du stock&\\
        FP $2.3$&\centering Déterminer la combinaison de microcapsules optimal pour délivrer la masse de produit donnée&\begin{itemize}
            \item Précision dans la masse délivrée : varie à chaque recette.
            \item Nombre maximale de microcapsules par réacteur : $5$
        \end{itemize}\\
        FP $2.4$&\centering Transmettre la position des microcapsules dans le stock et sur la plaque&\\
        FP $2.5$&\centering Informer le stock des microcapsules prélevées&\\
        FC $1$&\centering Respecter les dimensions de l'endroit confiné &Dimension de la boîte  : $\numproduct{133 x 95 x 94}~\unit{\cm}$\\
        FC $2$&\centering Utiliser les énergies disponibles& \begin{itemize}
            \item Électrique : \begin{itemize}
                \item $\qty{400}{\volt}$ triphasé
                \item $\qty{230}{\volt}$ monophasé
            \end{itemize}
            \item Pneumatique : $\qty{8}{\bar}$
        \end{itemize}\\ 
        FC $3$&\centering Utiliser les plaques déjà présentes&\begin{itemize}
            \item Nombre de trou sur la plaque de prise : $384$
            \item Nombre de trou sur la plaque de dépose : $48$
        \end{itemize}\\
        FC $4$&\centering S'adapter aux éléments déjà présents&\begin{itemize}
            \item API : Beckhoff
        \end{itemize}\\
        FC $5$&\centering Dimension des microcapsules : $\varnothing~\qty{3}{\mm}$&

    \end{longtable}
\rowcolors{0}{white}{white}