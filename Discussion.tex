\section{Points forts}
La réalisation de ce travail de Bachelor a permis de concevoir et d'implémenter un système robotisé pour la manipulation de \glspl{microcapsule} dans un environnement confiné. Bien que les objectifs initiaux aient été globalement atteints, certaines limitations et axes d'amélioration méritent d'être discutés.
\begin{itemize}
    \item \textbf{Précision et efficacité :} Le système robotisé mis en place s'est révélé capable de manipuler les \glspl{wellplate} avec une précision satisfaisante. L'intégration du robot avec les outils développés permis, théoriquement, d'automatiser efficacement le transfert des capsules entre les \glspl{wellplate} et les réacteurs.
    \item \textbf{Flexibilité du système :} La configuration retenue, avec un robot positionné au centre de la \glspl{glovebox}, offre une grande modularité, permettant de l'adapter à d'autres processus futurs au laboratoire SwissCat$+$.
    \item \textbf{Optimisation logicielle :} L'algorithme de recombinaison a démontré sa capacité à maximiser le nombre de recettes réalisables en un temps limité, grâce à une approche rigoureuse basée sur des techniques d'optimisation sous contraintes.
\end{itemize}
\section{Limitations}
\begin{itemize}
    \item \textbf{Prise des microcapsules :} La saisie des \glspl{microcapsule} n'est pas encore faisable, car le réseau pneumatique encore n'est pas encore développé, étant donné que les objectifs de la \gls{glovebox} n'est pas encore complètement définie.
    \item \textbf{Problème de manipulation :} Certains risques liés à la manipulation, comme le renversement des \glspl{wellplate}, des \glspl{paradox} ou la perte de \glspl{microcapsule} lors du transport, ont été identifiés. Des solutions ont été proposées (cf. \autoref{tab:solutions_risques}), leur implémentation reste partielle voir inexistante et nécessiterait une validation plus approfondie.
    \item \textbf{Temps de traitement : } Bien que l'algorithme permette une recombinaison efficace des \glspl{microcapsule}, lors de problèmes trop complexe, la recombinaison pourrait devenir un goulot d'étranglement. L'évaluation d'un autre algorithme, notamment en utilisant une méthode heuristique plus avancée pourrait améliorer les performances. La parallélisation des calculs améliorerait nettement les performances du système.
\end{itemize}
\section{Perspectives}
\begin{itemize}
    \item \textbf{Amélioration matérielle :} L'amélioration de la pince pour pouvoir saisir la \gls{paradox} (un entraxe plus grand suffirait).
    \item \textbf{Dévelopement du réseau pneumatique :} Le dévelopement d'un réseau pneumatique permettant au robot de faire le plus de déplacements possible est nécessaire.
    \item \textbf{Validation expérimentale :} Des tests à plus grande échelle avec des lots de \glspl{microcapsule} et des recettes variées permettront d'évaluer plus précisément les performances du système en conditions réelles.
    \item \textbf{Algorithme et intelligence artificielle :} L'intégration d'un algorithme d'apprentissage pour sélectionner présélectionner les \glspl{microcapsule} permettrait de diminuer le temps de traitement de l'algorithme.
\end{itemize}