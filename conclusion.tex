Ce projet de Bachelor a permis de dévelloper un système robotisé innovant pour la recombinaison des \glspl{microcapsule} chimiques dans un laboratoire de recherche. Les contributions principales incluent la conception et l'integration d'un robot dans une \gls{glovebox}, et la création d'un algorithme d'optimisation permettant de maximiser le nombre de recette réalisable.

L'ensemble du système répond aux éxigences initiales initiales en matières de précision, de fiabilité et de sécurité, tout en offrant une grande flexibilité pour des applications futures.Les résultats obtenus démontrent que l'automatisation de la recombinaison peut considérablement accélérer le processus expérimental, réduisant les erreurs humaines et augmentant l'efficacité global.

Cependant certains points, comme le réseau pneumatique, n'ont pas pu être réalisée à cause d'un manque de temps.

Des axes d'amélioration existent. Le temps de calcul de l'algorithme, la robustesse des outils et la gestion des risques liés à la manipulation sont des aspects à améliorer. L'ajout d'un module d'intelligence artificielle et l'amélioration des outils physique pourrait augmenter le rendement de la recombinaison.

En conclusion, ce travail de représente une avancée vers l'automatisation complète des tâches dans le laboratoire.

\vfil
\hspace{8cm}\makeatletter\@author\makeatother\par
\hspace{8cm}\begin{minipage}{5cm}
\end{minipage}