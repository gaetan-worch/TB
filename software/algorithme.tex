\section{Algorithme de recombinaison}
\subsection{Spécification de l'algorithme}
\subsubsection*{Objectif}
L'algorithme a pour objectif de sélectionner les \glspl{microcapsule}, dans le stock, à utiliser pour chaque recette d'un batch, soit environ $300$ recettes.

\subsection*{Contraintes}
Les contraintes de l'algorithme sont les suivantes : 
\begin{enumerate}
    \item Maximiser le nombre de recettes réalisées.
    \begin{itemize}
        \item Se situer entre dans la plage de tolérance de quantité pour chaque produit; 
        \item respecter la quantité maximale de \glspl{microcapsule} de chaque réacteur.
    \end{itemize}    
    \item Minimiser le nombre de \glspl{microcapsule} utilisées par réacteur.
\end{enumerate}
\subsection*{Entrées}
Il possède en entrée : 
\begin{itemize}
    \item les recettes du batch;
    \item les \glspl{microcapsule} présentes dans le stock;
    \item le nombre maximal de \gls{microcapsule} par réacteur.
\end{itemize}

\subsection*{Sorties}
Les sorties de l'algorithme incluent : 
\begin{itemize}
    \item Un tableau contenant pour chaque recette un liste avec les ids des capsules à utilisées ave le numéros de recettes sous la forme :
     $[([id_1, id_2, \dots, id_n], n^{\circ} recette_1), \dots, ([id_1, id_2, \dots, id_n], n^{\circ} recette_m)]$ avec $m$ le nombre de recette;
    \item Un tableau contenant la quantité de chaque produit pour chaque recette sous la forme :$[({"produit_1" : q_{produit_1}, \dots,  "produit_n" : q_{produit_n}}, n^{\circ} recette_1), \dots, \\({"produit_1" : q_{produit_1}, \dots,  "produit_n" : q_{produit_n}}, n^{\circ} recette_m)]$
    \item Le numéro des recettes qui ne sont pas réalisable
    \item Les éléments manquants pour réaliser les recettes non réalisable avec la forme : $[({"Produit_{manquant}~1" : q_{produit manquant 1}, \dots "Produit_{manquant}~n" : q_{produit_{manquant} n}"}, n^{recipe 1}),\\ \dots, ({"Produit_{manquant}~1" : q_{produit manquant 1}, \dots "Produit_{manquant}~n" : q_{produit manquant n}"}, n^{recipe m})]$ 
    \item Les éléments manquants pour réaliser les recettes non réalisable avec la forme : $[({"Produit_{manquant}~1" : q_{produit manquant 1}, \dots "Produit_{manquant}~n" : q_{produit manquant n}"}, n^{recipe 1}), \\ \dots, ({"Produit_{manquant}~1" : q_{produit manquant 1}, \dots "Produit_{manquant}~n" : q_{produit manquant n}"}, n^{recipe m})]$ 
\end{itemize}
\subsection{Définition du problème}
Le problème consite à trouver une combinaison de \glspl{microcapsule} pour chaque recette qui maximise le nombre de recettes réalisées.
\begin{equation}
    \max\left(\sum_{i} \text{RecetteRealisée}_i\right)
    \label{eq:objectif_algorithme}
\end{equation}
\subsubsection{Maximisation du nombre de recettes réalisées}
Avec :
\begin{itemize}
    \item $N$, le nombre de \glspl{microcapsule} dans le stockage, $N \in \mathbb{N}^*$;
    \item $k$, le nombre moyen de \glspl{microcapsule} par recette, $k\in \mathbb{N}^*$;
    \item $R$, le nombre de recette réalisable, $R\in \mathbb{N}^*$.
\end{itemize}
Le nombre théorique de recettes réalisable est :
\begin{equation}
   R_{th} = \left\lfloor \frac{N}{k}\right\rfloor
   \label{eq:nbre_recipe_th}
\end{equation}
L'équation (cf.\autoref{eq:objectif_algorithme}), implique une maximisation du nombre théorique de recettes réalisables (cf. \autoref{eq:maximisation}).
\begin{align}
    \max \left(\sum_{i}\text{RecetteRealisée}_i\right) &\implies \max\left(R_{th}\right) 
    \label{eq:maximisation} \\
    \max(R_{th}) = \max\left( \left\lfloor \frac{N}{k}\right\rfloor\right) &\implies \max\left(N\right) \vee \min\left(k\right)
\end{align}
Or, \(N\) est constant, donc :
\begin{equation}
    \max\left( \sum_{i} \text{RecetteRealisée}_i \right) \implies \min\left(k\right)
    \label{eq:min_k}
\end{equation}
À des fins de faciliter, l'utilisation de la minimisation de $k$ (cf. \autoref{eq:min_k}) sera préférée.
\subsubsection{Contraintes}
La liste des contraintes pour l'optimisateur sont les suivantes :
\begin{itemize}
    \item La quantité dans chaque réacteur doit être comprises dans la plage souhaitée;
    \item le nombre de \glspl{microcapsule} dans un réacteur doit être inférieur ou égal à la quantité maximal de \glspl{microcapsule} dans le réacteur;
    \item une \glspl{microcapsule} ne peut être utilisée plusieurs fois.
\end{itemize}
\subsubsection{\textit{Knapsack problem}}
\begin{quotation}
    \og The knapsack problem (KP) can be formally defined as follows: We are given an
    instance of the knapsack problem with item set N, consisting of n items j with profit
    Pj and weight Wj, and the capacity value c. (Usually, all these values are taken from
    the positive integer numbers.) Then the objective is to select a subset of N such
    that the total profit of the selected items is maximized and the total weight does not
    exceed c.\fg (\cite[p. 2]{KnapsackProblemsBook})
\end{quotation}
Le problème posé ressemble au \textit{knapsack problem}, cependant étant donné qu'il y a plusieurs réacteurs (l'équivalent du sac) le problème est donc plutôt un \textit{Multiple knapsack problem} \footnote{\parencite[p. 285]{KnapsackProblemsBook}}. Il y a encore une nuance entre le problème posé et un \textit{Multiple knapsack problem}, c'est que dans un réacteur, il peut y avoir plusieurs produits dans un seul réacteur. Donc pour chaque batch, il y a plusieurs problème du type \textit{Multiple knapsack problem}.
\subsection{Méthode d'optimisation}
\subsubsection{Optimisation générale}
L'optimisation générale consiste à traiter chaque produit séparément avec certaines contraintes (cf. \autoref{fig:algorithme_optimisateur}).
\begin{figure}[H]
    \centering
    \includegraphics[width=9cm]{assets/figures/diagramme_flux_solver.drawio}
    \caption{Algorithme général de l'optimisateur.}
    \label{fig:algorithme_optimisateur}
\end{figure}

\subsubsection{Optimisateur}
L'approche la plus intuitive pour optimiser le problème consiste à calculer toutes les combinaisons possibles puis de sélectionner la solution qui répond le mieux aux critères définis.

Le nombre de combinaisons $C$ possibles pour $k$ \glspl{microcapsule} et un stock $n$, se calcul comme suit :
\begin{equation}
    C_{k,n} = \frac{n!}{k!\cdot(n-k)!}
    \label{eq:combinaison}
\end{equation}
\begin{equation}
    C_{n} = \sum_{k = 1}^{l} C_{k,n} = \sum_{k=1}^{l}\frac{n!}{k!\cdot (n-k)!}
    \label{eq:nbre_combinaisons}
\end{equation} 
\begin{figure}[H]
    \centering
    \includesvg[width=\textwidth]{assets/figures/Software/nbre_combinaison.svg}
    \caption{Nombre de combinaisons possible en foction de la taille des réacteurs}
    \label{fig:nbre_combinaisons}
\end{figure}

\begin{figure}[H]
    \centering
    \begin{subfigure}{0.5\textwidth}
        \centering
        \includesvg[width=\textwidth]{assets/figures/Software/temps_seconde.svg}
        \caption{Temps d'execution en seconde}
    \end{subfigure}\hfill
    \begin{subfigure}{0.5\textwidth}
        \centering
        \includesvg[width=\textwidth]{assets/figures/Software/temps_annee.svg}
        \caption{Temps d'execution en année}
    \end{subfigure}
    \caption{Temps de calcul en fonction de la taille des réacteurs}
    \label{fig:tps_calc_combinaison}
\end{figure}
La \autoref{fig:tps_calc_combinaison} montre le temps nécessaire pour calculer toutes les combinaisons possibles (en prenant en compte le nombre de combinaisons possible (cf. \autoref{fig:nbre_combinaisons}) et le nombre des calcul par secondes moyen pour les ordinateurs en $2020$ environ $10^{11}$\footcite{petite_analyse_nbre_calculs_par_sec} opérations par secondes). Il est possible d'observer que le temps nécessaire pour les combinaisons dépassant $x$ \glspl{microcapsule} maximal par réacteur, devient des durées non concevable pour cette application.
Cette approche, bien qu'elle trouve toujours la solution optimale et qu'elle soit facilement compréhensible, n'est pas adapté au projet.

Plusieurs algorithmes existes pour résoudre ce type de problème : 
\begin{itemize}
    \item glouton;
    \item programmation dynamique;
    \item optimisation linéaire;
    \item heuristique;
    \item \textit{branch and cut};
    \item optimisation linéaire en nombres entiers;
    \item un algorithme génétique.
    \label{list:liste_algorithme}
\end{itemize}

Le problème peut être interpreté comme un problème contraint d'entier (\textit{Constraints Integer Problems} (CIPs)), car la sélection des \glspl{microcapsule} se fait de manière binaire (une \gls{microcapsule} est sélectionner ou non). Pour les CIPs, il existe des frameworks (notamment SCIP (\textit{Solver Consraint Integer Programs})) utilisant certains des algorithmes citées précédemment \autoref{list:liste_algorithme}.

L'utilisation de SCIP se fait avec la forme de l'optimisation avec contraintes :
\begin{align*}
    &\min\quad &x \\
    &\text{subject to}\quad &\sum_{i}\left( a_ix_i\right) \leq b \\
    &\text{and}\quad &x \in \mathbb{N}
\end{align*}
ou plus généralement : 
\begin{align*}
    &\min\quad &x \\
    &\text{subject to}\quad &Ax \leq b \\
    &\text{and}\quad &x \in \mathbb{N}
\end{align*}
Avec $b$ un vecteur et $A$ une matrice. Pour utiliser l'optimisateur, il faut définir la fonction de coût (cf. \autoref{subsubsection:fonction_de_cout}) à minimiser, et les contraintes (cf. \autoref{subsubsection:contraintes}).

\subsubsection{Matrice de décision}
\rowcolors{0}{white}{white}
L'optimisateur doit utilisé un vecteur de décision pour optimiser l'utilisation des \glspl{microcapsule}. Pour une recette le vecteur $\overrightarrow{x}$ est définie :
\begin{equation}
    x_{i}\in \left\{0, 1\right\}, \forall i\in\left\{1, 2, \dots, n\right\} \text{ avec } n\text{ le nombre de microcapsules en stock.}
    \label{eq:vecteur_decision_1r}
\end{equation}
Ce vecteur (\autoref{eq:vecteur_decision_1r}) est valable pour $1$ seul recette. Idéalement pour plusieurs recettes, il faudrait une matrice $X$ (cf. \autoref{eq:decision_matrix}), qui est par la suite \og vectorisée \fg afin d'obtenir le vecteur de décision final (cf. \autoref{eq:decision_vector_final}). 
\begin{equation}
    X = \left[
        \begin{array}{cccc}
            x_{0, 0} & x_{0, 1} & \cdots & x_{0, n} \\
            x_{1, 0} & x_{1, 1} & \cdots & x_{1, n} \\
            \vdots   & \vdots   & \ddots & \vdots \\
            x_{m, 0} & x_{m, 1} & \cdots & x_{m, n}
        \end{array}
        \right]
    \label{eq:decision_matrix}
\end{equation}
\begin{equation}
    \begin{split}
        \overrightarrow{x} = &\left[x_{0, 0}, \cdots, x_{0, n}, x_{1, 0}, \cdots, x_{1, n}, x_{m, 0}, \cdots, x_{m, n}, \right],\\
        &\text{ avec n le nombre de microcapsules, et m le nombre de recettes.}
    \end{split}
    \label{eq:decision_vector_final}
\end{equation}
\subsubsection{Matrices de contraintes}\label{subsubsection:contraintes}
Pour définir les deux matrices de contraintes ($A$ et $b$), il faut commencer par définir $b$ car la structure de $A$ en dépendra.
$b$ est un vecteur colonne qui comprends pour chaque recette :
\begin{enumerate}
    \item La quantité maximal souhaitée (noté $Q_{max}$ suivi du numéro de recette).
    \item La quantité minimale souhaitée (noté $Q_{min}$ suivi du numéro de recette).
    \item Le nombre maximal de \gls{microcapsule} par réacteur (noté $l$).
\end{enumerate}
Puis, s'en suit une colonne de $n$ ligne de $1$, correspondant aux nombre de fois qu'une \gls{microcapsule} peut être utilisé.
\begin{equation}
    b = \left[
        \begin{array}{c}
            Q_{max}1\\
            -Q_{min}1\\
            l\\
            \vdots\\
            Q_{max}n\\
            -Q_{min}n\\
            l\\
                1 \\
                \vdots \\
                1
        \end{array}
    \right]
\end{equation}
En sachant que $Ax \leq b$, il est possible d'en déduire que $A$ sera décomposée en sous matrices (des matrices identité $I_n$ et une autre matrice nommées $m_1$).
\begin{equation}
    A = \left[\begin{array}{cccc}
        m_1    & 0       & \cdots & 0\\
        0      & m_1     & \cdots & 0\\
        \vdots & \vdots  & \ddots & \vdots \\
        0      & 0       & \cdots &  m_1 \\
        I_n    & I_n     & \cdots &  I_n
    \end{array}\right]
\end{equation}
Pour la composition de $m_1$, la première ligne sert à déterminé si la somme de la quantité des capsules sélectionner est inférieur à la quantité limite, la deuxième à déterminé si la somme de  quantité des \glspl{microcapsule} sélectionnées est supérieur à la quantité minimale et la dernière ligne est présente pour verifier que la somme des \glspl{microcapsule} sélectionner ne dépasse pas la limite de \gls{microcapsule} maximum par réacteur.
\begin{equation}
    m_1 = \left[\begin{array}{cccc}
        Q_1  & Q_2  & \cdots & Q_n\\
        -Q_1 & -Q_2 & \cdots & -Q_n\\
        1    & 1    & \cdots & 1
    \end{array}\right]
\end{equation}
\subsubsection{Fonction de coût}\label{subsubsection:fonction_de_cout}
L'objectif de l'optimisation est de réduire le nombre de \glspl{microcapsule} utilisé, avec $x$ le vecteur de décision, le fonction de coût $f(\overrightarrow{x} )$ est donc : 
\begin{equation}
    f\left(\overrightarrow{x}\right) = \sum \overrightarrow{x} 
    \label{eq:prem_cost_function}
\end{equation}
Cependant en utilisant la fonction (cf. \autoref{eq:prem_cost_function}), si une seul des recettes du batch n'est pas réalisable, l'optimisateur retournera le fait que le problème n'est résoluble, sans fournir les résutat des recettes dont il a trouvé des solutions. Pour résoudre ce problème, l'ajout d'une \textit{slack variable}, nommé $z$ est indispensable. Cette variable prendra la quantité manquante de certaines recette non réalisable. Afin de ne pas tomber dans l'utilisation excesive de $\overrightarrow{z}$, il est nécessaire de rendre le coût de celle-ci plus imporante grâce à un ratio $\alpha$. La fonction de coût définitive devient :
\begin{equation}
    f\left(\overrightarrow{x}\right) = \sum\left(\overrightarrow{x} + \alpha \overrightarrow{z} \right)
    \label{eq:cost_function}
\end{equation}
Avec $\alpha$ définit arbitrairement à $10^{4}$.