\section{Algorithme de recombinaison}
\subsection{Spécification de l'algorithme}
\subsubsection*{Objectif}
L'algorithme a pour objectif de sélectionner les microcapsules, dans le stock, à utiliser pour chaque recette d'un batch, soit environ $300$ recettes.

\subsection*{Contraintes}
Les contraintes de l'algorithme sont les suivantes : 
\begin{enumerate}
    \item Maximiser le nombre de recettes réalisées.
    \begin{itemize}
        \item Respecter les quantités délivrées;
        \item respecter la quantité maximale de microcapsules de chaque réacteur.
    \end{itemize}    
    \item Minimiser le nombre de microcapsules utilisées par réacteur.
\end{enumerate}
\subsection*{Entrées}
Il possède en entrée : 
\begin{itemize}
    \item les recettes du batch;
    \item la position et la quantité de produit des microcapsules disponibles dans le stock;
    \item la tolérance des recettes;
    \item le nombre maximal de microcapsule par réacteur.
\end{itemize}

\subsection*{Sorties}
Les sorties de l'algorithme incluent : 
\begin{itemize}
    \item les numéros de plaque à utiliser;
    \item la position des microcapsules sur les plaques (sous la forme \og{}ligne, colonne\fg{} );
    \item les coordonnées et les numéros d'identification des réacteurs dans lesquels les microcapsules doivent être placées;
    \item les recettes non réalisables, avec indication des éléments manquants;
    \item les recettes réalisées.
\end{itemize}
\subsection{Méthode d'optimisation et stratégie d'ordonnancement}