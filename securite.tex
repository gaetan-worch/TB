La sécurité est un aspect important de ce projet, en raison de l'utilisation d'un robot UR$3$ dans un espace confiné, sous azote, pour manipuler des composants chimiques qui ne doivent pas entrer en contact, car ces éléments peuvent être réactifs entre eux. Cette section détaille les mesures mises en \oe{}uvre ou à mettre en \oe{}uvre pour assurer la sécurité des opérateurs, du matériel et des échantillons.
\section{Identification des risques}\label{sec:identification_risque}
Pour faire cette analyse, une approche ascendante\footnote{L'approche ascendante consiste à partir de la base (qu'est-ce qui peut être dangereux) puis remonter pour en trouver les effets. Inversement à l'approche descendante qui cherche les causes à partir d'un accident.} avec une matrice de risques (cf. \autoref{tab:ananylse_risque}) est utilisées. Pour la matrice de risque, la criticité du risque est le produit de la gravité et de l'occurrence de ce risque. Cette criticité servira à prioriser l'ordre de réduction des risques (plus la criticité est élevée, plus le risque est important et doit être réduit).
\rowcolors{3}{gray!10}{white}
\begin{landscape}
    \begin{table}[ht]
        \begin{tabular}{l|l|l|c|c|c}
        \multicolumn{1}{c|}{Id} & \multicolumn{1}{|c|}{Description}              & \multicolumn{1}{c|}{Impact}                                                                                                                    & \begin{tabular}[c]{@{}c@{}}Gravité\\ {[}1-5{]}\end{tabular} & \begin{tabular}[c]{@{}c@{}}Occurence\\ {[}1-5{]}\end{tabular} & \begin{tabular}[c]{@{}c@{}}Criticité\\ {[}1-25{]}\end{tabular} \\
        \hline
        1                      & Déchirement des gants par le robot              & Perte de l'étanchéité                                                                                                                          & 5                                                           & 1                                                             & \cellcolor{yellow} 5                                                              \\
        2                      & \gls{microcapsule} non saisie             & Erreur lors des analyses                                                                                                                       & 1                                                           & 2                                                             & \cellcolor{green}  2                                                              \\
        3                      & \gls{microcapsule} non déposée            & \begin{tabular}[c]{@{}l@{}}Erreur lors des analyses\\ Détérioration des prochaines \glspl{microcapsule}\end{tabular}                            & 3                                                           & 2                                                             & \cellcolor{yellow} 6                                                              \\
        4                      & Perte de \gls{microcapsule} lors du déplacement & \begin{tabular}[c]{@{}l@{}}Erreur lors des analyses\\ Contamination de la \gls{glovebox}\end{tabular}                                          & 4                                                           & 3                                                             & \cellcolor{orange} 12                                                             \\
        5                      & Renversement d'une \gls{wellplate}                    & \begin{tabular}[c]{@{}l@{}}Perte de la totalité des capsules de la \gls{wellplate} \\ Contamination de la \gls{glovebox}\end{tabular}                 & 5                                                           & 2                                                             & \cellcolor{orange} 10                                                             \\
        6                      & Renversement d'une \gls{paradox}                & \begin{tabular}[c]{@{}l@{}}Perte de la totalité des capsules présentes dans la \gls{paradox}\\ Contamination de la \gls{glovebox}\end{tabular} & 5                                                           & 3                                                             & \cellcolor{red}    15                                                             \\
        7                      & Collision entre les gants et le robot           & Contusions                                                                                                                                     & 5                                                           & 1                                                             & \cellcolor{yellow} 5                                                             
        \end{tabular}
        \caption{Analyses des risques}
        \label{tab:ananylse_risque}
        \end{table}
\end{landscape}

\section{Solutions}
Cette section à
a pour but de donner des solutions envisageables aux risques déterminés précédemment (cf. \autoref{sec:identification_risque}). Les solutions trouvées doivent impacter la gravité, le nombre d'occurrences ou les deux.
Par exemple, en prenant le risque à traiter en priorité (le numéro $6$ qui possède une criticité de $15$ (cf. \autoref{tab:ananylse_risque})), il est possible de modifier la \gls{paradox} afin d'y ajouter des ergots et ainsi améliorer le maintient de celles-ci dans la pince du robot, réduisant le nombre d'occurrences de ce risque. Mais il est également possible de rajouter un couvercle par-dessus la \gls{paradox} afin de ne pas éparpiller le contenu dans la \gls{glovebox}, réduisant ainsi la gravité du risque.
Toutes les solutions proposées n'ont pas forcément été appliquées lors du travail de Bachelor.

\begin{table}[H]
    \begin{tabular}{l|c|c}
        Id du risque & Solutions proposées                                                                                                                                          &\begin{tabular}[c]{@{}c@{}}Action{[}Gravité,\\Occurence ou les deux{]}\end{tabular}\\
        \hline
        $6$          & \begin{tabular}[c]{@{}l@{}}Améliorer l'ergonomie de la \gls{paradox} par rapport à la pince.\\Placer un couvercle sur la \gls{paradox} lors des déplacements.\end{tabular} & \begin{tabular}[c]{@{}l@{}}Occurence\\Gravité\end{tabular}\\
        $4$          & \begin{tabular}[c]{@{}l@{}}Augmenter la pression pour l'aspiration.\\ Mettre un bac récupérateur sous la pince.\end{tabular}                                   & \begin{tabular}[c]{@{}l@{}} Occurence\\ Gravité\end{tabular}\\
        $5$          & \begin{tabular}[c]{@{}l@{}}Ajout de trous sur la \gls{wellplate}\\pour améliorer la prise de la pince.\end{tabular}                                        & Occurence\\
        $3$          & Ajouter un système de soufflerie lors de la dépose.                                                                                                            & Occurence\\
        $7$          & Mettre des gants plus épais.                                                                                                                                   & Gravité\\
        $1$          & Arrondir tous les angles présents sur la pince.                                                                                                                & Occurence\\
        $2$          & Augmenter la pression pour l'aspiration.                                                                                                                       & Occurence
    \end{tabular}
    \caption{Solutions face aux risques (dans l'ordre de priorité)}
    \label{tab:solutions_risques}
\end{table}